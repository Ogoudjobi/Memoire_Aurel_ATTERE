\section{Limites et perspectives d'Apache Griffin}

La documentation d'Apache Griffin fournit une feuille de route sur les prochaines fonctionnalit\'es \`a d\'evelopper ainsi que celles disponibles. N\'eanmoins, certaines fonctionnalit\'es actuellement disponibles, pr\'esentent les limites suivantes: 
\begin{description}[parsep=0cm,itemsep=0cm]
\item[la gestion des mesures] : En effectuant des op\'erations sur l'interface utilisateur, on ne peut que cr\'eer, supprimer et mettre \`a jour que trois types de mesures. Les appels par l'\acrshort{api}, offrent \`a l'utilisateur un panel plus vari\'e de mesures (accuracy, profiling, timeliness, uniqueness ou distinctness, completeness, publish metrics). Toutefois l'absence d'impl\'ementation de connecteur de type \acrshort{jdbc} au niveau du web service limites assez la vari\'et\'e des sources;

\item[la gestion de l'ex\'ecution des mesures (jobs)]: L'utilisateur ne peut créer, modifier, supprimer et  planifier des jobs pour les donn\'ees venant par \textit{batch}, uniquement que pour les mesures \'elabor\'ees pour les connexions de type HIVE, AVRO, CUSTOM, via Postman. 
\end{description}
De plus, la mesure de type Uniqueness pr\'esente quelques dysfonctionnements et la Publish Metrics n'est pas document\'ee, faute de quoi nous n'avons pas pu la tester. Pour cette derni\`ere nous soupçonnons une obsolescence de la fonctionnalit\'e. Car m\^eme dans le code source elle s'est av\'er\'ee absente. Notons \'egalement que la documentation sur github bien que d\'etaill\'ee, n'est pas tenue \`a jour. 
\\

\`A court-terme, Apache Griffin envisage de supporter plus de sources de donn\'ees. Notamment un enrichissement de la connexion aux bases de donn\'ees relationnelles, ainsi qu'une connexion \`a Elasticsearch. De plus, Griffin pr\'evoit prendre en charge de mani\`ere native d'autres dimensions de qualit\'e de donn\'ees comme la coh\'erence et la validit\'e. Pour l'instant, il est laiss\'e \`a la charge de l'utilisateur la d\'efinition des attentes vis-\`a-vis de ces dimensions. Une nouvelle fonctionnalit\'e est \'egalement cibl\'ee:  il s'agit de la d\'etection d'anomalie en analysant les m\'etriques calcul\'ees par Elasticsearch. Notons \'egalement que dans le cadre de cette \'etude, nous avons utilis\'e la version 0.6.0 d'Apache Griffin, mais durant cette exploration, de nouvelles fonctionnalités et des am\'eliorations de celles existantes ont \'et\'e publi\'ees sur github. Il s'agit de la réécriture et de la red\'efinition des diff\'erentes dimensions. Ainsi la nouvelle version s'appuiera probablement sur les dimensions suivantes :  l'Accuracy, la Completeness, la Duplication et le Profiling. En plus, de ces quatre dimensions Griffin intègre une nouvelle: la conformit\'e du sch\'ema. Cette mesure \'evalue si oui ou non les donn\'ees respectent le type qu'on leur attribue. Il est question l\`a de l'\'evaluation de la validit\'e des donn\'ees.  Ces modifications dans les mesures se traduisent aussi par une réécriture de la configuration du fichier \textit{.json} ainsi que de la pr\'esentation des r\'esultats. En termes de connectivit\'e, une classe pour la connexion \`a Elasticsearch a \'egalement \'et\'e ajout\'ee. Ce qui annonce de belles perspectives pour le d\'eveloppement de l'outil. 
\\

