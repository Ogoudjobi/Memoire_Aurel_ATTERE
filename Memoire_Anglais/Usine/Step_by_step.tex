\documentclass[12pt,a4paper]{article}
\usepackage[utf8x]{inputenc}
\usepackage{ucs}
\usepackage[francais]{babel}
\usepackage[T1]{fontenc}
\usepackage{amsmath}
\usepackage{amsfonts}
\usepackage{amssymb}
\usepackage{graphicx}
\author{Aurel ATTERE}
\begin{document}
\section{Cadre contextuel}
\subsection{Qualité de données dans le secteur de l'assurance }
Il comportera la détection et correction des problèmes de qualité dans le cadre du projet socle de données de la Digital Factory  :\\
	
Volet outils\\
Connexion des outils Griffin et Great\_expectations aux différentes bases de données de l'équipe data\\
Comparaison, choix et implémentation de l'outil le plus adapté\\	

Volet qualité de donnée\\
Revue des différentes données du socle et détection des incohérences par rapport aux différentes sources de données \/ extractions utilisées par le métier (visualisation des résultats sur PowerBI desktop)\\
Priorisation des champs à mettre en qualité en urgence\\
Implémentation d'algorithmes de redressement des données quand cela est possible \\
Injection de d'open data quand cela est possible pour une mise en qualité des données priorisées
\end{document}