\cleardoublepage
\phantomsection
\addcontentsline{toc}{chapter}{Abstract}
\section*{ABSTRACT}

In the age of big data, the quality of any decision depends on the quality of the data used. Indeed, without reliable data, a company can potentially make bad decisions. Our study aims, first, to detect with Apache Griffin the quality problems on the data of the Digital Factory of Saham Maroc, then to apply corrective measures when possible. From the theoretical analysis carried out, we have retained that data quality refers to the ability of all the intrinsic characteristics of the data (completeness, consistency, uniqueness, validity, accuracy, timeliness) to meet internal requirements and external requirements to the organization. To be able to evaluate it efficiently, we used Apache Griffin. It is a big data platform, which offers data quality services. It allows editing quality rules and store metrics for visualization or historical purposes. The quality assessment of the 'Detail\_Victime' and 'Inventaire\_Sinistre' extraction tables revealed several inconsistencies that we corrected using PySpark, by means of a rectification algorithm.\\
\textbf{Key words}: quality, data, big data, evaluation, correction, Griffin, inconsistencies