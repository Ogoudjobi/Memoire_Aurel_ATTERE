\documentclass[12pt,a4paper, fleqn]{report}
\usepackage[utf8]{inputenc}
\usepackage[T1]{fontenc}
\usepackage{graphicx}
\usepackage{fullpage}
\usepackage{eso-pic}
\usepackage{amsmath}
\usepackage[Glenn]{fncychap}%Glenn Rejne* Conny Sonny* Lenny Bjarne
\usepackage{amsfonts}
\usepackage{amssymb}
\usepackage{listings}
\usepackage{lmodern}
\usepackage{color}
\usepackage[absolute]{textpos}

\definecolor{darkWhite}{rgb}{0.94,0.94,0.94}

\lstset{
  aboveskip=3mm,
  belowskip=-2mm,
  backgroundcolor=\color{darkWhite},
  basicstyle=\footnotesize,
  breakatwhitespace=false,
  breaklines=true,
  captionpos=b,
  commentstyle=\color{red},
  deletekeywords={...},
  escapeinside={\%*}{*)},
  extendedchars=true,
  framexleftmargin=16pt,
  framextopmargin=3pt,
  framexbottommargin=6pt,
  frame=tb,
  keepspaces=true,
  keywordstyle=\color{blue},
  language=Python,
  literate=
  {²}{{\textsuperscript{2}}}1
  {⁴}{{\textsuperscript{4}}}1
  {⁶}{{\textsuperscript{6}}}1
  {⁸}{{\textsuperscript{8}}}1
  {€}{{\euro{}}}1
  {é}{{\'e}}1
  {è}{{\`{e}}}1
  {ê}{{\^{e}}}1
  {ë}{{\¨{e}}}1
  {É}{{\'{E}}}1
  {Ê}{{\^{E}}}1
  {û}{{\^{u}}}1
  {ù}{{\`{u}}}1
  {â}{{\^{a}}}1
  {à}{{\`{a}}}1
  {á}{{\'{a}}}1
  {ã}{{\~{a}}}1
  {Á}{{\'{A}}}1
  {Â}{{\^{A}}}1
  {Ã}{{\~{A}}}1
  {ç}{{\c{c}}}1
  {Ç}{{\c{C}}}1
  {õ}{{\~{o}}}1
  {ó}{{\'{o}}}1
  {ô}{{\^{o}}}1
  {Õ}{{\~{O}}}1
  {Ó}{{\'{O}}}1
  {Ô}{{\^{O}}}1
  {î}{{\^{i}}}1
  {Î}{{\^{I}}}1
  {í}{{\'{i}}}1
  {Í}{{\~{Í}}}1,
  morekeywords={*,...},
  numbers=left,
  numbersep=10pt,
  numberstyle=\tiny\color{black},
  rulecolor=\color{black},
  showspaces=false,
  showstringspaces=false,
  showtabs=false,
  stepnumber=1,
  stringstyle=\color{gray},
  tabsize=4,
  title=\lstname,
} 
 
\newcommand{\HRule}{\rule{\linewidth}{0.5mm}}
\newcommand{\blap}[1]{\vbox to 0pt{#1\vss}}
\newcommand\AtUpperLeftCorner[3]{%
  \put(\LenToUnit{#1},\LenToUnit{\dimexpr\paperheight-#2}){\blap{#3}}%
}
\newcommand\AtUpperRightCorner[3]{%
  \put(\LenToUnit{\dimexpr\paperwidth-#1},\LenToUnit{\dimexpr\paperheight-#2}){\blap{\llap{#3}}}%
}
 
\title{\Large{FORMATION : INITIATION AUX M\'ETIERS DE LA DATA}}
\author{\textsc{Inner Planning Engineers} 
 \vspace*{1cm} 

{\textsc{DEV'TEAM}} \\  \textbf{\textsc{Junior Data Science Consulting (JDSC)}} \\ \vspace*{1cm} 
    
 }
\date{}

\makeatletter
 

\begin{document}



\ClearShipoutPicture
%\tableofcontent
\chapter*{PLAN V1}

DEDICACES\\
REMERCIEMENTS\\
SOMMAIRE\\
LISTE DES TABLEAUX\\
LISTE DES FIGURES\\
SIGLES ET ABREVIATIONS\\
AVANT-PROPOS\\
RESUME -ABSTRACT\\
INTRODUCTION GENERALE\\
PARTIE 1: PRESENTATION DE L'ENVIRONNEMENT DE STAGE 
\begin{itemize}
\begin{itemize}
\begin{itemize}
\item[Chapitre 1:] Pr\'esentation de la structure d'accueil
\item[Chapitre 2:] Pr\'esentation du sujet d'\'etude
\begin{enumerate}
\item Contexte
\item Cahier de charges
\item Pr\'esentation de l'existant \`a Saham
\item Chronogramme
\end{enumerate}
\end{itemize}
\end{itemize}
\end{itemize}
PARTIE 2: ANALYSE CONCEPTION ET CHOIX TECHNOLOGIQUE
\begin{itemize}
\begin{itemize}
\begin{itemize}
\item[Chapitre 1:] Cadre conceptuel de la qualit\'e
\item[Chapitre 2:] Revue empirique et technologique
\item[Chapitre 3:] Choix technologique
\end{itemize}
\end{itemize}
\end{itemize}
PARTIE 3: DEPLOIEMENT ET EXPLOITATION DE LA SOLUTION
\begin{itemize}
\begin{itemize}
\begin{itemize}
\item[Chapitre 1:] D\'eploiment de Apache Griffin
\item[Chapitre 2:] D\'etection et correction des probl\`emes de qualit\'e
\end{itemize}
\end{itemize}
\end{itemize}
CONCLUSION GENERALE

\chapter*{PLAN}
DEDICACES\\
REMERCIEMENTS\\
SOMMAIRE\\
LISTE DES TABLEAUX\\
LISTE DES FIGURES\\
SIGLES ET ABREVIATIONS\\
AVANT-PROPOS\\
RESUME -ABSTRACT\\
INTRODUCTION GENERALE\\
\textbf{Chapitre 1 :}\\

1.Cadre contextuel\\
1-1 Qualité de données dans un secteur de l'assurance (une sorte de problématique)\\
1-2 Présentation de l'organisme d’accueil\\

2- Méthodologie de recherche\\                                          
2-1 : Objectifs de la recherche\\
2-2:  Questions de recherche\\
2-3 : Méthodes de la recherche\\
2-4 : Valeur du travail (intérêts de l’étude)\\

3- Etude de l'existant\\
3-1 : Analyse de l'existant\\
3-2 : Etude du besoin\\

4- Management du projet\\                                                                 
4-1 Objectif du projet\\
4-2 Choix de la méthode Agile\\
4-3 : Organisation du projet\\
4-4 : Planning du projet\\ \\
\newpage
.\\
\textbf{Chapitre 2:} \\

1-Background (On trouvera un nom plus esthétique)

2-Etat de l'art\\ \\
\textbf{Chapitre 3:}\\ 

1- Cadre conceptuel de la qualité

2- Revue empirique et technologique

3- Choix technologique\\ \\
\textbf{Chapitre 4:}\\

1- Déploiement de Apache Griffin

2- Détection et correction des problèmes de qualité\\
CONCLUSION GENERALE\\
BIBLIOGRAPHIE\\
ANNEXE\\
\end{document}