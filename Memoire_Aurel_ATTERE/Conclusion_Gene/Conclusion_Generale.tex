\cleardoublepage
\phantomsection
\addcontentsline{toc}{part}{CONCLUSION G\'EN\'ERALE}

\vspace*{0.5cm}
\section*{\centering \Huge Conclusion G\'en\'erale}
\titlerule[2.0pt]
\vspace{1cm}
Le problème crucial que pose la qualité des données \`a la \acrlong{df}, nous a amené \`a mettre en place Apache Griffin, un outil d'\'evaluation de la qualit\'e des donn\'ees, afin de pouvoir d\'etecter les anomalies dans les donn\'ees du socle, pour par la suite proposer des algorithmes de correction. Afin de mener \`a bien cette t\^ache, nous nous sommes attel\'es \`a comprendre au pr\'ealable le cadre d'analyse th\'eorique de la qualit\'e. Cette compr\'ehension \`a par ricochet favoriser la compr\'ehension de l'outil Apache Griffin, et permet une meilleure compr\'ehension de ses fonctionnalit\'es. \\

Face aux exigences en termes de qualit\'e et de service de la \acrshort{df}, nous avons mis en œuvre nos connaissances techniques afin de pouvoir adapter l'outil aux besoins formul\'es. Cette adaptation a permis une \'evaluation efficace de la qualit\'e des donn\'ees du socle et favoris\'e la mise en \'evidence de plusieurs irr\'egularit\'es suivant les dimensions suivantes: Compl\'etude, Unicit\'e, Validit\'e et Coh\'erence. La suite du projet consistait \`a apporter des correctifs aux anomalies d\'etect\'ees. Ce qui nous a amen\'e \`a \'ecrire en PySpark des algorithmes de fiabilisation des donn\'ees entachées d'irrégularités. La visualisation des m\'etriques sur le tableau de bord r\'ealis\'e \`a cet effet, permet aux d\'ecideurs de pouvoir suivre l'\'evolution de la qualit\'e des diff\'erentes donn\'ees au fil du temps.\\

La prochaine \'etape du volet qualit\'e dans la politique de gouvernance des donn\'ees du socle, doit inclure la d\'efinition claire et nette des m\'etriques et des seuils de m\^eme que l'int\'egration de l'outil dans leur architecture. La gestion de la qualit\'e doit donc commencer par un plan et une strat\'egie bien document\'ee. Ce qui va permettre de configurer les outils pour \'evaluer la qualit\'e. Les r\'esultats obtenus, permettront de mesurer l'ampleur des problèmes de qualité des données et de déterminer si l'acquisition des données et le processus d'analyse sont adéquats. Enfin, la dernière composante qu'est l'amélioration de la qualité, doit permettre l'\'epuration des donn\'ees, la compr\'ehension de la non qualit\'e et la prise de mesures pour améliorer le processus. Ce qui permettra d'ajuster le plan et ainsi de suite.