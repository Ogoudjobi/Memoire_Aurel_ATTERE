\section{Cadre contextuel}

\subsection{Pr\'esentation de l'organisme d'accueil}
\subsubsection{\textbf{Sanlam Group}}
Fondée en 1918 en tant que compagnie d’assurance vie, Sanlam (South African National Life Assurance Company Limited) est aujourd’hui un groupe leader des services financiers diversifiés, basé en Afrique du Sud. Il déploie ses activités à travers l’ensemble du continent africain ainsi qu’en Malaisie, aux Etats-Unis, en France, en Suisse, en Inde et en Australie. Groupe financier de référence coté à la bourse de Johannesburg, Sanlam offre des solutions financières complètes et personnalisées dans tous les segments du marché, à travers ses cinq (5) pôles d’activités : Sanlam Personal Finance, Sanlam Pan-Africa, Sanlam Investments, Sanlam Corporate et Santam (South African National Trust and Assurance Company Limited).
\begin{figure}[!h]
  \caption{Organigramme de Sanlam Group}  \label{fig:xray}
  \begin{center}
  \hspace*{-1.0in}
  \includegraphics[scale=0.30]{Static/Organigramme.png} 
  \end{center}
\end{figure}
\\

C'est l'un des plus grands groupes d'assurance ayant une couverture internationale, dans le monde, en termes de présence, avec une présence directe et indirecte dans quarante-quatre (44) pays, à l'exception de l'Afrique du Sud. Sanlam Group, fournit plus précisément des solutions et produits financiers aux particuliers et aux entreprises \`a savoir :
\begin{itemize}[parsep=0cm,itemsep=0cm]
\item la planification financière et conseil;
\item l'assurance vie et non vie, la r\'eassurance;
\item la gestion de patrimoine, interm\'ediaire boursier;
\item la gestion des fonds (pensions, retraites,...).
\end{itemize}
%\vspace {0.5cm}
%\paragraph{}

\`A  Sanlam Group, on développe une vision portée sur la création de valeur pour le client, placé au cœur de toute stratégie de développement. Par ailleurs, il ambitionne de consolider ses positions sur le segment des solutions d’investissement sur les marchés développés. 
%Sa vision strat\'egique est de créer de la valeur durable pour toutes les parties prenantes. Elle se d\'ecline sur 3 axes principaux:
%\begin{itemize}[parsep=0cm,itemsep=0cm]
%\item pour l'Afrique du Sud : \^etre leader dans la gestion de patrimoine, le management et la protection;
%\item pour l'Afrique, l'Inde, la Malaisie et le Liban : \^etre un Groupe de services financiers panafricain de premier plan avec une présence significative en Inde et en Malaisie;
%\item pour les march\'es d\'evelopp\'es : se positionner sur la niche de gestion du patrimoine et de placements sur des marchés développés spécifiques.
%\end{itemize}
%\vspace {0.5cm}
Au sein du Groupe, notre stage s'est d\'eroul\'e au p\^ole Sanlam Pan-Africa. 
\subsubsection{\textbf{Sanlam Pan-Africa}}
\acrfull{spa} est le pôle d’activités de Sanlam Group opérant sur les marchés émergents (hors Afrique du Sud). \acrshort{spa} assure ainsi le développement et le déploiement d’une gamme diversifiée de produits d’assurance Vie et Non-Vie, ainsi que des solutions d’investissement, des services bancaires et de crédit à la consommation, la bancassurance, la gestion d’actifs et des produits d’assurances Non-Vie spécialisées pour l’Afrique, l’Inde, la Malaisie et le Liban. L’Afrique est aujourd’hui une composante fondamentale de la vision de Sanlam Group, d’où la mission stratégique fondamentale dévolue à \acrshort{spa} : ériger un groupe de services financiers panafricain de premier plan. Disposant de la première empreinte panafricaine qui lui garantit un rayonnement continental unique, \acrshort{spa} se positionne notamment en tant que partenaire privilégié des multinationales et autres réseaux de distribution. Au sein de SPA, nous avons effectu\'e notre stage \`a la \acrfull{df} de Saham Maroc, afin de b\'en\'eficier d'un suivi ad\'equat.

\subsubsection{\textbf{Digital Factory }} 
La \acrlong{df}, est un espace où des experts en digital travaillent et r\'efl\'echissent sur l’accélération de la transformation digitale et l’élaboration de solutions pour une expérience client optimale. La \acrshort{df} se veut \textit{customer centric} et proactive pour livrer des produits efficaces et utiles
pour les métiers en un temps record. Pour cela, en plus des compétences recrutées et de leurs expériences, elle mise sur une approche organisationnelle du travail : l’agilité à travers la méthode Scrum\footnote{concept d\'efini \`a les section 1.2.3}. Grâce à la \acrlong{df}, Saham Assurance Maroc innove sur des produits qu’elle lance au fur et à mesure. Elle comprend plusieurs \'equipes constituées aussi bien de d\'eveloppeurs, d'\acrfull{ui_ux} designers, de \textit{data scientists}, \textit{data engineers}, \textit{data architects}, de coach agile...
\\

En raison des conditions sanitaires, notre stage s'est effectu\'e physiquement \`a Colina Participations (qui est la \textit{holding} du groupe Sanlam d\'etenant les parts dans les diff\'erentes filiales en Afrique hors Maroc et Afrique du Sud) mais virtuellement (\`a distance) \`a la \acrlong{df}. Nous avons \'et\'e suivis et encadr\'es dans un environnement convivial et agile. Ce stage nous a permis d'\^etre beaucoup plus autonome et de pouvoir transformer les besoins m\'etiers ainsi que les contraintes \textit{business} en r\'ealisations techniques.  


\subsection{Qualit\'e des donn\'ees en assurance}
L'assurance est un secteur qui a pour principale mission de fournir une prestation lors de la survenance d'un événement incertain et aléatoire souvent appelé 'risque'. La prestation, généralement financière, peut être destinée à un individu, une association ou une entreprise, en échange de la perception d'une cotisation ou prime. L'assurance se d\'efinit aussi comme \'etant une op\'eration par laquelle l'assur\'e transf\`ere ses risques \`a l'assureur en contrepartie du paiement d'une prime. 
\\

Bien qu'\'etant un secteur majeur dans les activit\'es \'economiques d'un pays, l'assurance se distingue par l'inversion du cycle de production. En effet, il est impossible aux compagnies de savoir avec certitude, combien la prestation qu’elles vendent leur coûtera, la prime étant payée par le client avant l'indemnisation. Ainsi, pour fixer le montant de sa prime ou calculer les provisions, l’assureur ne peut se baser que sur des études statistiques lui permettant de se faire une idée de combien lui coûtera sa prestation en analysant par exemple le taux de sinistralité et le co\^ut moyen des sinistres ant\'erieurs. Cela ne donne pas pour autant la certitude qu’il n’aura pas à faire face à des sinistres majeurs (en termes de fr\'equence ou de co\^ut). La fiabilité des systèmes d’information et la qualité des données doivent alors constituer un objectif permanent, dans un environnement  incertain et malgr\'e tout concurrentiel.  \\
 

Ce sujet est donc au cœur du modèle d’affaires des assureurs. La parfaite maîtrise des systèmes d’information et des dispositifs de sécurité sont des leviers stratégiques voire vitaux pour maintenir une position de leader dans le domaine. Elle permet notamment de  : 

\begin{itemize}[parsep=0cm,itemsep=0cm]
\item se distinguer en termes de tarification, gr\^ace à une segmentation plus efficiente avec la possibilité de créer de nouvelles offres ;
\item se distinguer en termes de gestion des risques, par une optimisation des couvertures et des provisions;
\item mettre en place de meilleurs systèmes de détection des fraudes.
\end{itemize}    
L’enjeu est crucial à tous les niveaux : que ce soit pour une bonne appréhension des risques, pour mener les études actuarielles, réaliser les tarifications, évaluer les provisions ou fiabiliser les modèles, etc. \\

Les organismes assureurs sont donc, sensibles aux gains de productivité espérés qui pourront se traduire dans la compétition avec les autres acteurs du marché. De ce fait, les défauts de qualité des données peuvent constituer un frein à la compétitivité du groupe face aux concurrents et s’avérer être coûteux pour plusieurs raisons. Tout d’abord, ils (les d\'efauts de qualit\'e) rendent plus difficiles l’ensemble des travaux de production puisqu’ils complexifient les traitements. De plus, des données de mauvaise qualité sont susceptibles de conduire à une dégradation ou à l’allongement des travaux et des analyses qui en résultent. Enfin, elles impactent la qualit\'e des services offerts aux clients. En effet, selon une étude du \acrfull{mit} (MIT, 2017) \cite{MIT2017}, la non-qualité occasionne une perte d'argent estimée entre 15\% et 25\% du chiffre d’affaires total de la plupart des entreprises. Environ 20 ans plus t\^ot, cette perte s'\'evalue \`a 5\% - 10\% du revenu des entreprises \cite{Techno7}. De m\^eme, dans \cite{efrontech},  l’institut Gartner estime que plus de 25\% des données des plus grandes entreprises mondiales sont erronées et précise \'egalement qu'un tel probl\`eme n’a pas une conséquence informatique mais plut\^ot une cons\'equence commerciale, se chiffrant en millions de devises mon\'etaires. Cette perte représentant environ un quart des revenus, s’explique par les mauvais choix stratégiques opérés à partir d’informations erronées, mais aussi par le temps perdu par les services informatiques à traiter ces données inexactes:  contr\^oles, corrections et maintenances. Ramener au secteur de l'assurance, une non-qualit\'e des donn\'ees peut nuire aux décisions prises s’agissant aussi bien des exigences r\'eglementaires que des choix strat\'egiques de l’entreprise (mauvaise interprétation de la situation actuelle par exemple) \cite{Axysweb_Consequences}. Prendre une décision à partir de mauvaises informations affecte l’entreprise, ses clients ou ses partenaires.
\\

La d\'efinition d'une bonne strat\'egie de gouvernance des données est un sujet qui prend de l’importance dans les entreprises et les administrations. Il urge alors de se pencher plus s\'erieusement sur cette probl\'ematique tout en prenant en compte le contexte technologique. Il est impensable que les données de l’assureur ne soient pas à la hauteur des attentes. Une défaillance sur ce volet-là a de lourdes conséquences. Point d'ancrage de toute strat\'egie de gouvernance des donn\'ees, la mise en place d'un projet de qualit\'e des donn\'ees requiert quelques interrogations: qu'est-ce que la qualit\'e des donn\'ees et comment la mesurer? Quels sont les diff\'erents outils de qualit\'e des donn\'ees et quels sont les avantages qu'offre Apache Griffin vis-\`a-vis de ces derniers? Comment corriger les probl\`emes de qualit\'e des donn\'ees d\'etect\'es?

