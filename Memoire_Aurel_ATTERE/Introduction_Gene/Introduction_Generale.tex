\cleardoublepage
\phantomsection
\addcontentsline{toc}{part}{INTRODUCTION G\'EN\'ERALE}
\pagenumbering{arabic}
\vspace*{0.5cm}
\section*{\centering \Huge Introduction G\'en\'erale}
\titlerule[2.0pt]
\vspace{1cm}
 %Pour les entreprises, c'est un moyen d'augmenter significativement leur profit sur une période donnée. Ainsi, \`a travers les innovations techniques qu'elles induisent, les diff\'erentes r\'evolutions industrielles ont profondément  transformé l'économie. [...] Bien que déjà visible dans ses réalisations comme l’intelligence artificielle, la nanotechnologie ou l’information quantique, nous en sommes qu’au début.%
%En \'economie, le progrès technique est considéré comme une source majeure de croissance surtout pour les entreprises. Ainsi, 
Le monde fait face aujourd'hui aux prémices d’une quatrième révolution industrielle favoris\'ee par l'essor du \textit{big data}. \`A travers les innovations techniques qu'elles induisent, les diff\'erentes r\'evolutions industrielles ont profondément transformé l'économie et significativement augment\'e le profit des entreprises. Ainsi, l'explosion du volume des donn\'ees de m\^eme que l'apparition et la vulgarisation de nouvelles technologies de stockage, de traitement et de mise en valeur des donn\'ees, ont suscité de profonds changements au sein des entreprises. Afin de profiter des externalit\'es de cette r\'evolution, les entreprises sont de plus en plus tourn\'ees vers des approches \textit{data-driven} et \textit{data-centric}. Cela se manifeste \`a travers une architecture dans laquelle les données constituent l'actif principal.  Aujourd'hui, les entreprises effectuent des analyses \textit{big data}, des modélisations de diagnostic et des traitements de données pour atteindre l'excellence sur le marché. Il va donc sans dire que toute entreprise souhaitant s'inscrire dans une dynamique \textit{data-driven} et \textit{data-centric} se doit d'avoir des donn\'ees de qualit\'e sur lesquelles s'appuyer. Dans le cas contraire, vous imaginez bien les décisions désastreuses qu'elle  pourrait prendre. Il est alors n\'ecessaire, voire primordial pour toute entreprise de se doter d'une solide strat\'egie de gouvernance des donn\'ees et donc de qualit\'e des donn\'ees.\\

Fort de ce constat, Saham Assurance Maroc \`a travers sa \acrlong{df}, a d\'ecid\'e de mettre en place dans le cadre de son projet socle de donn\'ees, une politique de gouvernance des donn\'ees. Cette derni\`ere, donne une attention particuli\`ere au volet qualit\'e. En effet, la qualit\'e des donn\'ees revêt un aspect très important pour les entreprises d'assurances, en ce sens qu'elle facilite grandement l'activit\'e de l'assureur dans la d\'etermination des risques, le calcul des tarifications et même la d\'etection des fraudes, sans oublier les exigences réglementaires du r\'egime prudentiel Solvabilit\'e 2. C’est dans ce contexte que le sujet suivant : \textit{« détection et correction des problèmes de qualité de données dans le cadre du projet socle de données »}, nous a été confié durant nos six mois de stage \`a la Digital Factory de Saham Assurance Maroc.\\

L'objectif principal de cette \'etude est de d\'etecter avec Apache Griffin les probl\`emes de qualit\'e de quelques tables du socle de donn\'ees, puis d'appliquer lorsque cela est possible des algorithmes de correction des anomalies d\'etect\'ees. Pour cela, nous avons suivi une m\'ethodologie en trois (3) grandes \'etapes : 
\begin{itemize}[parsep=0cm,itemsep=0cm]
    \item l'installation et la configuration d'Apache Griffin afin de r\'epondre aux exigences en terme de plateforme de qualit\'e de donn\'ees;
    \item l'utilisation d'Apache Griffin pour la d\'etection des anomalies et incoh\'erences dans les donn\'ees du socle et;
    \item la proposition d'algorithmes de correction.
\end{itemize}
Avant d'aborder et d'appliquer cette m\'ethodologie, nous allons dans un premier chapitre situer l'\'etude dans son cadre contextuel en pr\'esentant l'entreprise ainsi que les d\'etails du projet. Le deuxi\`eme chapitre  quant \`a lui permettra de mieux comprendre les réflexions théoriques sous-jacentes. Ces deux premiers chapitres constituent la premi\`ere partie du pr\'esent m\'emoire. Dans une seconde partie, nous procéderons à la description détaillée en deux chapitres \'egalement, des r\'esultats obtenus en appliquant la m\'ethodologie ci-dessus \'enonc\'ee. 


